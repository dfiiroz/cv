\begin{rubric}{Research Experience}
\subrubric{BSS Monograph}
\entry*[] \textbf{Title of the Paper: } Effect of Economic Growth and Inflation on Unemployment: An Empirical Analysis in Bangladesh from 1991 to 2019
%
\entry*[] \textbf{Abstract: } This paper analyzes the effect of economic growth and inflation on unemployment, considering the time-series data period from 1991 to 2019 in the case of Bangladesh. The unemployment rate is increasing alarmingly, which can lead to instability, and it is a major challenge that policymakers must face. In this research monograph, the Augmented Dickey-Fuller Test (ADF) and Phillips Perron (PP) were applied to test the unit root of the variable. The Autoregressive Distributed Lag (ARDL) model, the Bounds test of cointegration, and the Error Correction Model (ECM) were used to look at the short-run and long-run relationship among the variables. The Granger Causality test was also used to determine whether there is a unidirectional or bidirectional cause-and-effect relationship among the variables. F-statistics is 9.89, which is greater than the upper bound I(0), indicating a long-run relationship. In the long run, all variables significantly impact unemployment, but only economic growth has a significant effect in the short run. In addition, evidence suggests a unidirectional Granger causal relationship among the variables. The study's diagnostic results reveal that the model does not contain heteroscedasticity, autocorrelation, and multicollinearity; there is normality and stability, and the model was correctly specified.

\end{rubric}